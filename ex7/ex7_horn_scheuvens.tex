\documentclass[10pt]{article}
\usepackage[utf8]{inputenc}

\usepackage{amssymb, amsmath, amsthm, amsfonts, algorithmic, algorithm, graphicx}
\usepackage{color}
\usepackage{bbm}
\usepackage[dvipsnames]{xcolor} 
\usepackage[colorlinks,linkcolor=blue,citecolor=blue]{hyperref}
\usepackage{array}
\usepackage{ifthen}
\usepackage{mathtools}
\usepackage[thinc]{esdiff}

\renewcommand{\baselinestretch}{1.1}
\setlength{\topmargin}{-3pc}
\setlength{\textheight}{8.5in}
\setlength{\oddsidemargin}{0pc}
\setlength{\evensidemargin}{0pc}
\setlength{\textwidth}{6.5in}

\newtheorem{theorem}{Theorem}[section]
\newtheorem{lemma}[theorem]{Lemma}
\newtheorem{proposition}[theorem]{Proposition}
\newtheorem{corollary}[theorem]{Corollary}
\newtheorem{question}[theorem]{Question}
\newtheorem{result}[theorem]{Result}
\newtheorem{definition}[theorem]{Definition}
\newtheorem{example}[theorem]{Example}
\newtheorem{remark}[theorem]{Remark}
\newtheorem{assumption}[theorem]{Assumption}
\numberwithin{equation}{section}

\def \endprf{\hfill {\vrule height6pt width6pt depth0pt}\medskip}
\renewenvironment{proof}{\noindent {\bf Proof} }{\endprf\par}

% Notational convenience,
% real numbers 
\newcommand{\R}{\mathbb{R}}  
% Expectation operator
\DeclareMathOperator*{\E}{\mathbb{E}}
% Probability operator
\DeclareMathOperator*{\Prob}{\mathbb{P}}
\renewcommand{\Pr}{\Prob}

% You may define additional macros here.
\newcommand{\unita}{\begin{pmatrix} 1\\0\end{pmatrix}}
\newcommand{\unitb}{\begin{pmatrix} 0\\1\end{pmatrix}}
\newcommand{\gramint}{\int_{-1}^1}
\newcommand{\limn}{\underset{n\rightarrow \infty}{lim}}
\newcommand{\epix}{e^{\pi-x}}


\begin{document}

\begin{center}
    \sc ML 4101: Mathematics for Machine Learning --- Fall 24
\end{center}

\noindent Friederike Horn \& Bileam Scheuvens

Justify all your claims.
\section*{Exercise 1 (Netwon Method vs. Gradient Descent)}

\subsection*{Solution}

\begin{enumerate}
\item[a)]{
    $$\nabla \frac{1}{2} x^t Qx - b^t x = Qx - b \overset{!}{=} 0$$
    $$ \Rightarrow Qx = b$$
    Divide by $Q$, given that $Q$ is symmetric and PD and therefore invertible.
    $$ x = Q^{-1}b$$
    This is a minimum and therefore the optimal solution to the problem, if the Hessian is positive definite.
    $$\nabla^2 \frac{1}{2} x^t Qx - b^t x = Q$$
    Since we know $Q$ to be PD, $x^* = Q^{-1}b$.
  }

\item[b)] {
    $$||x_{k+1} - x^*|| = ||x_k - \alpha \nabla f(x_k) - x^* ||$$
    $$ = ||(x_k - x^*)  - \alpha (Qx_k + b) ||$$
    Since $b = Qx^*$
    $$= ||(x_k - x^*) - \alpha Q (x_k + x^*) ||$$
    $$= ||(x_k - x^*) (I - \alpha Q)  ||$$
$$\leq ||(x_k - x^*)||\cdot ||(I - \alpha Q)  ||$$
    Let $\alpha = \frac{2}{\lambda_{max} + \lambda_{min}}$
    $$\leq ||(x_k - x^*)||\cdot ||(I - \frac{2Q}{\lambda_{max} + \lambda_{min}})  ||$$
$$= ||(x_k - x^*)||\cdot \left(\frac{\lambda_{max} - \lambda_{min}}{\lambda_{max} + \lambda_{min}}\right)$$

  }
\item[c)] {
    $$x_{k+1} = x_k - Q^{-1}\nabla (\frac{1}{2} x_k^t Qx - b^t x_k)$$
  $$x_{k+1} = x_k - Q^{-1}(Qx_k - b)$$
  $$x_{k+1} = x_k - Q^{-1}Qx_k + Q^{-1}b$$
  $$x_{k+1} = x_k - Ix_k + Q^{-1}b$$
  $$x_{k+1} = Q^{-1}b$$
  $$x^* = Q^{-1}b$$
}
\end{enumerate}
\section*{Exercise 2 (Convexity and Continuity)}

\subsection*{Solution}
\begin{enumerate}
\item[a)]{
    Imagine y to be any point between x and z, analogous to how continuity is defined in general.
    We know that:
    $$f(tx + (1-t)z) < tf(x) + (1-t)f(z)$$
    Let $y = tx + (1-t)z$. Then:
    $$ f(y) < tf(x) + (1-t)f(z)$$ 
    Subtract $f(x)$
    $$ f(y) - f(x) < (t-1)f(x) + (1-t)f(z)$$
    $$ f(y) - f(x) < (t-1)(f(x) - f(z))$$
    Divide by $y-x$, since $x < y$ and therefore $x-y \neq 0$
    $$ \frac{f(y) - f(x)}{y-x} < \frac{(t-1)(f(x) - f(z))}{y-x}$$
    Choose $t = \frac{y-x}{z-x} +1$
    $$ \frac{f(y) - f(x)}{y-x} < \frac{f(x) - f(z)}{z-x}$$
    The argument is analogous for the slope between $z-x$ to $z-y$, concluding the proof.
}
\item[b)]{

}
\item[c)]{
    $f$ is not necessarily continous in $a$ and $b$ as exemplified by:
    $$f(x) = 
    \begin{cases}
      x^2 & \text{if } x \in [a,b]\\
      0 & \text{otherwise}
    \end{cases}
    $$

  }

\end{enumerate}

\section*{Exercise 3 (Convexity)}
\subsection*{Solution}

\begin{enumerate}
\item[a)]{

}
\item[b)]{
    Consider the logarithm. $log$ is concave, since its second derivative $-\frac{1}{x^2}$ is negative.
    Since the logarithm is monotone it preserves inequality and we can instead consider
    $$log(\frac{1}{n} \sum_{i=1}^{n}x_i) \leq log((\prod_{i=1}^n x_i)^{\frac{1}{n}})$$
    $$log(\frac{1}{n} \sum_{i=1}^{n}x_i) \leq \frac{1}{n} log(\prod_{i=1}^n x_i)$$
    $$log(\frac{1}{n} \sum_{i=1}^{n}x_i) \leq \frac{1}{n} \sum_{i=1}^n log(x_i)$$
    This holds by the result from a).

}
\end{enumerate}

\section*{Exercise 4 (Dual Problem)}
\subsection*{Solution}
\begin{enumerate}
\item[a)]{
    $$ f(x, y) = max\;\; x+y$$
    $$\text{subject to } x^2 +2y^2 \leq 5$$
    Bring to normal form by minimizing and bringing constraint to the form $\leq 0$
    $$ f(x) = min\;  -x-y$$
    $$\text{subject to } x^2 + 2y^2 - 5 \leq 0$$
    $$\mathcal{L}(x,y) = min\; -x-y - \lambda (x^2 + 2y^2 -5)$$
    Solve for stationarity: 
    $$\nabla \mathcal{L} \overset{!}{=} 0$$
    $$\diffp{L}{x} = 1 + 2\lambda x$$
    $$\Rightarrow x  = \frac{1}{2\lambda}$$
    $$\diffp{L}{y} = 1 + 4\lambda y$$
    $$\Rightarrow y  = \frac{1}{4\lambda}$$
    Solve for primal feasibility:
    $$x^2 + 2y^2 -5 \leq 0$$
    $$\frac{1}{4\lambda^2} + 2\frac{1}{16\lambda^2} -5 \leq 0$$
    $$\frac{1}{4\lambda^2} + \frac{1}{8\lambda^2} -5 \leq 0$$
    $$\frac{3}{8\lambda^2}  \leq 5$$
    $$\frac{3}{8}  \leq 5\lambda^2$$
    $$\frac{3}{40}  \leq \lambda^2$$
    $$\sqrt{\frac{3}{40}} \leq \lambda$$
    $$\Rightarrow x = \frac{1}{2\sqrt{\frac{3}{40}}} = \frac{1}{\sqrt{\frac{4  \cdot 3}{40}}}= \sqrt{\frac{10}{3}}$$
    $$ \Rightarrow y = \sqrt{\frac{5}{3}}$$
    The constraint is active since $\lambda > 0$ (dual feasibility).
    This makes intuitive sense since the function would otherwise be unbounded and the optimal solution would be $x = y = \infty$.
}
\item[b)]{
    The Lagranian of the problem is:
    $$L(x,\lambda) = min \; -c^t x - \lambda (Ax-b)$$
    Therefore we obtain the dual function $g(\lambda) = \underset{x}{inf} \mathcal{L}(x,\lambda)$.
    

}
\end{enumerate}
\end{document}
