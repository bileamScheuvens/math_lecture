\documentclass[10pt]{article}
\usepackage[utf8]{inputenc}

\usepackage{amssymb, amsmath, amsthm, amsfonts, algorithmic, algorithm, graphicx}
\usepackage{color}
\usepackage{bbm}
\usepackage[dvipsnames]{xcolor} 
\usepackage[colorlinks,linkcolor=blue,citecolor=blue]{hyperref}
\usepackage{array}
\usepackage{ifthen}
\usepackage{mathtools}

\renewcommand{\baselinestretch}{1.1}
\setlength{\topmargin}{-3pc}
\setlength{\textheight}{8.5in}
\setlength{\oddsidemargin}{0pc}
\setlength{\evensidemargin}{0pc}
\setlength{\textwidth}{6.5in}

\newtheorem{theorem}{Theorem}[section]
\newtheorem{lemma}[theorem]{Lemma}
\newtheorem{proposition}[theorem]{Proposition}
\newtheorem{corollary}[theorem]{Corollary}
\newtheorem{question}[theorem]{Question}
\newtheorem{result}[theorem]{Result}
\newtheorem{definition}[theorem]{Definition}
\newtheorem{example}[theorem]{Example}
\newtheorem{remark}[theorem]{Remark}
\newtheorem{assumption}[theorem]{Assumption}
\numberwithin{equation}{section}

\def \endprf{\hfill {\vrule height6pt width6pt depth0pt}\medskip}
\renewenvironment{proof}{\noindent {\bf Proof} }{\endprf\par}

% Notational convenience,
% real numbers 
\newcommand{\R}{\mathbb{R}}  
% Expectation operator
\DeclareMathOperator*{\E}{\mathbb{E}}
% Probability operator
\DeclareMathOperator*{\Prob}{\mathbb{P}}
\renewcommand{\Pr}{\Prob}

% You may define additional macros here.
\newcommand{\unita}{\begin{pmatrix} 1\\0\end{pmatrix}}
\newcommand{\unitb}{\begin{pmatrix} 0\\1\end{pmatrix}}
\newcommand{\gramint}{\int_{-1}^1}
\newcommand{\limn}{\underset{n\rightarrow \infty}{lim}}
\newcommand{\epix}{e^{\pi-x}}


\begin{document}

\begin{center}
    \sc ML 4101: Mathematics for Machine Learning --- Fall 24
\end{center}

\noindent Friederike Horn \& Bileam Scheuvens

Justify all your claims.
\section*{Exercise 1 (Extremal points)}
Consider the function $f: \mathbb{R}^2 \to \mathbb{R}; (x, y) \mapsto x^3 + 1/3y^3 -12x -y$
\begin{enumerate}
\item[a)]{
$(x, y)$ is an extremal point if $\frac{\partial f}{\partial x}=0$ and $\frac{\partial f}{\partial y} = 0$. \\
$$\frac{\partial f}{\partial x} = 3 x^2 -12$$
$$\frac{\partial f}{\partial y} = y^2 -1$$

From this we can get four different extremal points:
$$(x_1, y_1) = (2, 1)$$
$$(x_2, y_2) = (-2, 1) $$
$$(x_3, y_3) = (2, -1) $$
$$ (x_4, y_4) = (-2, -1) $$
In order to classify them we need to compute the Hessian: 
$$
H = \begin{pmatrix} 6x & 0 \\
0 & 2y \end{pmatrix}.
$$
We directly see that this is diagonal and thus the eigenvalues correspond to the diagonal entriess \\
We see that  for $(x_1, y_1)$ we have positive eigenvalues and therefore a strict  local minimum. Likewise, for $(x_4, y_4)$ we have negative eigenvalues and therefore a strict local  maximum. For the other two points the Hessian matrix has a negative and a positive eigenvalue and thus it is indefinite and we have two saddle points.}
\item[b)] {
(x, y) is a global maximum iff there exists no other $(x', y')$ such that $f(x', y') > f(x, y))$ and likewise for global minimum. \\
In this case the function has no global maximum or minimum. E.g. $f(x_1, y_1) = 8 + 1/3 -24 -1 = -17.3$, but we find that $f(-3, 0) = -27 < -17.3$. \\
Similarly, $f(x_4, y_4) = -8 - 1/3 + 24 +1 = 16.67$, but $f(3, 0)= 27 > 16.67 = f(x_4, y_4)$. 
}
\item[c)] {
Consider $g: \mathbb{R}^3 \to \mathbb{R}, (x, y, z) \mapsto \alpha x^2e^y + y^2 e^z + z^2 e^x$, with $\alpha \in \mathbb{R}$. 
The point $(0, 0, 0)$ is an extremal point (local, minimum, maximum or saddle point) iff the first derivative is zero i.e. $\nabla f = \textbf{0}$.
$$
\nabla f = \begin{pmatrix} \alpha 2x e^y + z^2 e^x\\ \alpha x^2 e^y + 2y e^z \\ y^2 e^z + 2z e^x \end{pmatrix}_{(0, 0)}
$$
$$
= \begin{pmatrix} 0 \\ 0 \\ 0 \end{pmatrix}
$$
To differentiate between the different points we compute the Hessian:
$$
H = {\begin{pmatrix}
\alpha 2 e^y + z^2e^x & \alpha 2x e^y & 2z e^x \\
\alpha 2x e^y  & x^2 e^y + 2 e^z & 2ye^z \\
2z e^x & 2y e^z & y^2 e^z + 2 e^x
\end{pmatrix}}_{(0, 0)}
$$
$$
\begin{pmatrix}
\alpha 2   &  0& 0  \\
0  & 2 & 0 \\
0 & 0 &  2 
\end{pmatrix}.
$$
As this is again a diagonal matrix we can directly read out the eigenvalues from the diagonal entries and as two af them are greater than zero we can only have a local minimum or saddle point. If $\alpha <0$ the matrix is indefinite and we therefore have a saddle point. If $a\geq 0$ we have a local minimum. 
}
\end{enumerate}
\section*{Exercise 2 (Derivatives)}

\begin{enumerate}
\item[a)]{
The directional derivatives exist in point $(0, 0, 0)$ iff the $\lim_{x \to 0+} \frac{\partial f}{\partial x}_{y=0} = \lim_{x \to 0-} \frac{\partial f}{\partial x}_{y=0}$
 and likewise for $y$. 
 We therefore bult the two derivatives:
$$
\frac{\partial f_+}{\partial x} = 
$$
}
\end{enumerate}

\section*{Exercise 3 (Taylor Series)}
Given the function $f: \mathbb{R}^2 \to \mathbb{R}; (x_1, x_2) \mapsto \frac{1}{1-x_1-x_2}$ find the Taylor series around point $(0, 0)$ and the set over which it converges. 
\begin{enumerate}
\item[Solution]{
We first note for the derivatives:
$$
\frac{\partial f^n}{{\partial x_1}^\alpha{\partial x_2}^{n-\alpha}} = n! \frac{1}{{(1 - x_1 - x_2)}^n}.  
$$
This is because the partial derivative in $x_1$ direction is equal to the derivative in $x_2$ direction and each derivative simply multiplies by the power of the denominator and then increases the power of the denominator by one. \\
For this reason the multivariate Taylor series (around the point $(0, 0)$) can immediately be written out as:
$$
T_f = \sum_{n=0}^\infty \sum_{\alpha=0}^n \frac{n!}{\alpha! (n-\alpha)!}x_1^\alpha x_2^{n-\alpha}.
$$
This can be rewritten as:
$$
T_f = \sum_{n=0}^\infty \sum_{\alpha=0}^n \binom{n}{\alpha} x_1^\alpha x_2^{n-\alpha} = \sum_{n=0}^\infty (x_1 + x_2)^n.
$$
This series converges iff $|x_1+x_2| < 1. 

\end{enumerate}
\end{document}
