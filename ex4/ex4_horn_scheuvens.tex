\documentclass[10pt]{article}
\usepackage[utf8]{inputenc}

\usepackage{amssymb, amsmath, amsthm, amsfonts, algorithmic, algorithm, graphicx}
\usepackage{color}
\usepackage{bbm}
\usepackage[dvipsnames]{xcolor} 
\usepackage[colorlinks,linkcolor=blue,citecolor=blue]{hyperref}
\usepackage{array}
\usepackage{ifthen}
\usepackage{mathtools}

\renewcommand{\baselinestretch}{1.1}
\setlength{\topmargin}{-3pc}
\setlength{\textheight}{8.5in}
\setlength{\oddsidemargin}{0pc}
\setlength{\evensidemargin}{0pc}
\setlength{\textwidth}{6.5in}

\newtheorem{theorem}{Theorem}[section]
\newtheorem{lemma}[theorem]{Lemma}
\newtheorem{proposition}[theorem]{Proposition}
\newtheorem{corollary}[theorem]{Corollary}
\newtheorem{question}[theorem]{Question}
\newtheorem{result}[theorem]{Result}
\newtheorem{definition}[theorem]{Definition}
\newtheorem{example}[theorem]{Example}
\newtheorem{remark}[theorem]{Remark}
\newtheorem{assumption}[theorem]{Assumption}
\numberwithin{equation}{section}

\def \endprf{\hfill {\vrule height6pt width6pt depth0pt}\medskip}
\renewenvironment{proof}{\noindent {\bf Proof} }{\endprf\par}

% Notational convenience,
% real numbers 
\newcommand{\R}{\mathbb{R}}  
% Expectation operator
\DeclareMathOperator*{\E}{\mathbb{E}}
% Probability operator
\DeclareMathOperator*{\Prob}{\mathbb{P}}
\renewcommand{\Pr}{\Prob}

% You may define additional macros here.
\newcommand{\unita}{\begin{pmatrix} 1\\0\end{pmatrix}}
\newcommand{\unitb}{\begin{pmatrix} 0\\1\end{pmatrix}}
\newcommand{\gramint}{\int_{-1}^1}


\begin{document}

\begin{center}
    \sc ML 4101: Mathematics for Machine Learning --- Fall 24
\end{center}

\noindent Friederike Horn \& Bileam Scheuvens

Justify all your claims.
\section*{Exercise 1 (Principal Component Analysis)}
\begin{enumerate}
\item[a)]{
Suppose $\lambda$ is an eigenvalue to the eigenvector $v$ of $\Sigma$.
$$\lambda \langle x, x\rangle = \langle \lambda x, x\rangle $$
$$=  \langle \Sigma x, x\rangle $$
Since $\Sigma = \bar{\Sigma}$
$$=  \langle  x, \Sigma x\rangle $$
$$=  \langle  x, \lambda x\rangle $$
$$= \bar{\lambda} \langle  x,  x\rangle $$
If $\lambda = \bar{\lambda}$ then $\lambda$ must be in $\mathbb{R}$.

  }
\item[b)]{
    By the Rayleigh coefficient, vector that maximizes the variance is the normed eigenvector $\frac{v_1}{||v_1||}$ corresponding to the largest eigenvalue $\lambda_1$ of $X^TX$. The resulting variance is $\sqrt{\lambda_1}$ i.e. the first singular value $\sigma_1$ of $X$.
  }
\item[c)]{
    Analogously, the vector orthogonal to $v_1$ that maximizes the variance is the normed eigenvector $\frac{v_2}{||v_2||}$ corresponding to the largest eigenvalue $\lambda_2$ of $X^TX$. The resulting variance is the second singular value $\sigma_2$ of $X$.
  }
\item[d)]{
  The k-th principal component is obtained by eigendecomposition of $X^TX$ and ordering the eigenvalues such that $\lambda_1 < \lambda_2 < \dots < \lambda_d$. Then the k-th principal component is $X*v_k$ with variance $\sqrt{\lambda_k}$.
  To obtain the lower dimensional representation $\tilde{X}$ we multiply the matrix with the eigenvectors of $XX^T$ as columns $\tilde{U}$ by the matrix with diagonal elements $\tilde{\sigma_i} = \sqrt{\lambda_i}$ $\tilde{\Sigma}$ and the matrix with eigenvectors of $X^TX$ as rows $\tilde{V}^T$.
  $$\tilde{X} = \tilde{U}\tilde{\Sigma}\tilde{V}^T$$
  }
\end{enumerate}

\section*{Exercise 2 (Singular Value Decomposition)}
\begin{enumerate}
\item[a)]{
    The singular values are equal to the ordered eigenvalues of $MM^T$ and $M^TM$.
    $$M^TM = \begin{bmatrix}-4&1&3\\ 2&2&2\end{bmatrix}\begin{bmatrix}-4&2\\ 1&2 \\3&2\end{bmatrix} = \begin{pmatrix}27 & 0\\ 0&12\end{pmatrix}$$
    Since the result is diagonal, the eigenvalues are easily determined as $\lambda_1=27, \lambda_2=12$ which results in the singular values $\sigma_1 = \sqrt{27}, \sigma_2 = \sqrt{12}, \sigma_3 = 0$.
  }
\item[b)]{
    The singular values of a matrix $A$ are the union of the roots of the eigenvalues of $A^TA$ and $AA^T$. Since the union is commutative, $eigs(AA^T) \cup eigs(A^TA) = eigs(A^TA) \cup eigs(AA^T)$, so the singular values are identical.
  }
\item[c)]{
    Note that $\Sigma * \Sigma^\# = I$ only if $\sigma_i \neq 0 \forall i$, otherwise it acts as $I$ on the subspace on nonzero singular values.
    \begin{itemize}
      \item{
        $$AA^\#A \overset{!}= A$$
        $$U\Sigma V^T V \Sigma^\# U^T U\Sigma V^T = A$$
        $$U\Sigma \Sigma^\# \Sigma V^T = A$$
        $$U \Sigma V^T = A$$
      }
    \item{
        $$A^\# A A^\# \overset{!}{=} A^\#$$
        $$V\Sigma^\# U^T U\Sigma V^T V\Sigma^\# U^T = A^\#$$
        $$V\Sigma^\# \Sigma \Sigma^\# U^T = A^\#$$
        $$V\Sigma^\# U^T = A^\#$$
  
      }
    \item{
        $$(A A^\#)^T \overset{!}{=} A A^\#$$
        $$(U\Sigma V^T V \Sigma^\# U^T)^T = A A^\#$$
        $$U\Sigma^\# V^T V \Sigma U^T = A A^\# $$
        $$U\Sigma^\# V^T V \Sigma U^T = U\Sigma V^T V \Sigma^\# U^T $$
        $$U\Sigma^\# \Sigma U^T = U\Sigma \Sigma^\# U^T $$
      }
    \item{
        $$(A^\# A)^T \overset{!}{=} A^\# A$$
        $$(V\Sigma^\# U^T U \Sigma V^T)^T = A^\# A$$
        $$V\Sigma U^T U \Sigma^\# V^T = A^\# A$$
        $$V\Sigma U^T U \Sigma^\# V^T = V\Sigma^\# U^T U \Sigma V^T $$
        $$V\Sigma \Sigma^\# V^T = V\Sigma^\# \Sigma V^T $$
      }
    \end{itemize}
}
\end{enumerate}
\section*{Exercise 3 (Condition Number)}
\begin{enumerate}
\item[a)]{
    First prove that $||Av|| \overset{!}{\leq}  ||A||_2 \cdot ||v||$ 
    $$||Av|| \leq \underset{x\neq 0}{max} \frac{||Ax||}{||x||} ||v||$$
    If $||v|| = 0$ this holds since $0 = ||A||_2 \cdot 0$. Otherwise we divide by $||v||$ to obtain:
    $$\frac{||Av||}{||v||} \leq \underset{x\neq 0}{max} \frac{||Ax||}{||x||}$$
    This clearly holds.
    


  }
\item[b)]{
  }
\end{enumerate}


\end{document}
